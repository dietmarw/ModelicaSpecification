\chapter{Modelica DAE Representation}\doublelabel{modelica-dae-representation}

In this appendix, the mapping of a Modelica model into an appropriate
mathematical description form is discussed.

In a first step, a Modelica translator transforms a hierarchical
Modelica simulation model into a ``flat'' set of Modelica
``statements'', consisting of the equation and algorithm sections of all
used components by:

\begin{itemize}
\item
  expanding all class definitions (flattening the inheritance tree) and
  adding the equations and assignment statements of the expanded classes
  for every instance of the model.
\item
  replacing all connect-equations by the corresponding equations of the
  connection set (see \ref{generation-of-connection-equations}).
\item
  mapping all algorithm sections to equation sets.
\item
  mapping all when-clauses to equation sets (see \ref{when-equations}).
\end{itemize}

As a result of this transformation process, a set of equations is
obtained consisting of differential, algebraic and discrete equations of
the following form where ( ):

and where

\begin{longtable}[]{|p{2cm}|p{12cm}|}
\hline \endhead
\textbf{p} & Modelica variables declared as \emph{parameter} or
\emph{constant}, i.e., variables without any
time-dependency.\\ \hline
\emph{t} & Modelica variable \emph{time}, the independent (real)
variable.\\ \hline
\textbf{x}(t) & Modelica variables of type \emph{Real}, appearing
differentiated.\\ \hline
\textbf{m}(t\textsubscript{e}) & Modelica variables of type
\emph{discrete Real, Boolean, Integer} which are unknown. These
variables change their value only at event instants t\textsubscript{e}.
pre(m) are the values of m immediately before the current event
occurred.\\ \hline
\textbf{y}(t) & Modelica variables of type \emph{Real} which do not fall
into any other category (= algebraic variables).\\ \hline
c\textbf{(}t\textsubscript{e}) & The conditions of all if-expressions
generated including when-clauses after conversion, see \ref{when-equations}).\\ \hline
\emph{relation}(\textbf{v}) & A relation containing variables
v\textsubscript{i}, e.g. v\textsubscript{1} \textgreater{}
v\textsubscript{2}, v\textsubscript{3} \textgreater{}= 0.\\ \hline

\end{longtable}

For simplicity, the special cases of the noEvent() operator and of the
reinit() operator are not contained in the equations above and are not
discussed below.

The generated set of equations is used for simulation and other analysis
activities. Simulation means that an initial value problem is solved,
i.e., initial values have to be provided for the states x, \ref{initialization-initial-equation-and-initial-algorithm}.
The equations define a DAE (Differential Algebraic Equations) which may
have discontinuities, a variable structure and/or which are controlled
by a discrete-event system. Such types of systems are called
\emph{hybrid DAEs}. Simulation is performed in the following way:

\begin{enumerate}
\item
  The DAE (1c) is solved by a numerical integration method. In this
  phase the conditions c of the if- and when-clauses, as well as the
  discrete variables m are kept constant. Therefore, (1c) is a
  continuous function of continuous variables and the most basic
  requirement of numerical integrators is fulfilled.
\item
  During integration, all relations from (1a) are monitored. If one of
  the relations changes its value an event is triggered, i.e., the exact
  time instant of the change is determined and the integration is
  halted. As discussed in \ref{events-and-synchronization}, relations which depend only on
  time are usually treated in a special way, because this allows to
  determine the time instant of the next event in advance.
\item
  At an event instant, (1) is a mixed set of algebraic equations which
  is solved for the Real, Boolean and Integer unknowns.
\item
  After an event is processed, the integration is restarted with 1.
\end{enumerate}

Note, that both the values of the conditions c as well as the values of
m (all discrete Real, Boolean and Integer variables) are only changed at
an event instant and that these variables remain constant during
continuous integration. At every event instant, new values of the
discrete variables m and of new initial values for the states x are
determined. The change of discrete variables may characterize a new
structure of a DAE where elements of the state vector x are
\emph{disabled}. In other words, the number of state variables,
algebraic variables and residue equations of a DAE may change at event
instants by disabling the appropriate part of the DAE. For clarity of
the equations, this is not explicitly shown by an additional index in
(1).

At an event instant, including the initial event, the model equations
are reinitialized according to the following iteration procedure:

\begin{lstlisting}[language=modelica]
    known  variables: x, t, p
    unkown variables: dx/dt, y, m, pre(m), c 

   // pre(m) = value of m before event occured
   loop
      solve (1) for the unknowns, with pre(m) fixed
      if m == pre(m) then break
      pre(m) := m    
   end loop 
\end{lstlisting}
Solving (1) for the unknowns is non-trivial, because this set of 
equations contains not only Real, but also Boolean and Integer unknowns.
Usually, in a first step these equations are sorted and in many cases
the Boolean and Integer unknowns can be just computed by a forward
evaluation sequence. In some cases, there remain systems of equations
(e.g. for ideal diodes, Coulomb friction elements) and specialized
algorithms have to be used to solve them.

Due to the construction of the equations by "flattening" a Modelica
model, the hybrid DAE (1) contains a huge number of sparse equations.
Therefore, direct simulation of (1) requires sparse matrix methods.
However, solving this initial set of equations directly with a numerical
method is both unreliable and inefficient. One reason is that many
Modelica models, like the mechanical ones, have a DAE index of 2 or 3,
i.e., the overall number of states of the model is less than the sum of
the states of the sub-components. In such a case, every direct numerical
method has the difficulty that the numerical condition becomes worse, if
the integrator step size is reduced and that a step size of zero leads
to a singularity. Another problem is the handling of idealized elements,
such as ideal diodes or Coulomb friction. These elements lead to mixed
systems of equations having both Real and Boolean unknowns. Specialized
algorithms are needed to solve such systems.

To summarize, symbolic transformation techniques are needed to transform
(1) in a set of equations which can be numerically solved reliably. Most
important, the algorithm of Pantelides should to be applied to
differentiate certain parts of the equations in order to reduce the
index. Note, that also explicit integration methods, such as Runge-Kutta
algorithms, can be used to solve (1c), after the index of (1c) has been
reduced by the Pantelides algorithm: During continuous integration, the
integrator provides x and t. Then, (1c) is a linear or nonlinear system
of equations to compute the algebraic variables y and the state
derivatives dx/dt and the model returns dx/dt to the integrator by
solving these systems of equations. Often, (1c) is just a linear system
of equations in these unknowns, so that the solution is straightforward.
This procedure is especially useful for real-time simulation where
usually explicit one-step methods are used.
