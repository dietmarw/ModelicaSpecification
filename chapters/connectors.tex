\chapter{Connectors and Connections}\doublelabel{connectors-and-connections}

This chapter covers connectors, connect-equations, and connections.

The special functions cardinality, rooted {[} \emph{deprecated}{]},
Connections.isRoot, and Connections.rooted may not be used to control
them.

\section{Connect-Equations and Connectors}\doublelabel{connect-equations-and-connectors}

Connections between objects are introduced by connect-equations in the
equation part of a class. A connect-equation has the following syntax:

\textbf{connect} "(" component-reference "," component-reference ")" ";"

The connect-equation construct takes two references to connectors
{[}\emph{a connector is an instance of a connector class}{]}, each of
which is either of the following forms:

\begin{itemize}
\item
  $c_1.c_2...c_n$,
  where $c_1$ is a connector of the class,
  n\textgreater{}=1 and $c_{i+1}$ is a connector element of
  $c_i$ for i=1:(n-1).
\item
  m.c, where m is a non-connector element in the class and c is a
  connector element of m.
\end{itemize}

There may optionally be array subscripts on any of the components; the
array subscripts shall be parameter expressions or the special operator
``:''. If the connect construct references array of connectors, the
array dimensions must match, and each corresponding pair of elements
from the arrays is connected as a pair of scalar connectors.

{[}\emph{Example of array usage}:

\begin{lstlisting}[language=modelica]
  connector InPort = input Real;
  connector OutPort = output Real;
  block MatrixGain
    input InPort u[size(A,2)];
    output OutPort y[size(A,1)];
    parameter Real A[:,:] = [1];
  equation
    y=A*u;
  end MatrixGain;
  Modelica.Blocks.Sources.Sine sinSource[5];
  MatrixGain gain (A = 5*identity(5));
  MatrixGain gain2(A = ones(2,5));
  OutPort x[2];
equation
  connect(sinSource.y, gain.u); // Legal
  connect(gain.y, gain2.u); // Legal
  connect(gain2.y, x); // Legal
\end{lstlisting}
{]}

The three main tasks are to:

\begin{itemize}
\item
  Elaborate expandable connectors.
\item
  Build connection sets from connect-equations.
\item
  Generate equations for the complete model.
\end{itemize}

\subsection{Connection Sets}\doublelabel{connection-sets}

A connection set is a set of variables connected by means of
connect-equations. A connection set shall contain either only flow
variables or only non-flow variables.

\subsection{Inside and Outside Connectors}\doublelabel{inside-and-outside-connectors}

In an element instance M, each connector element of M is called an
outside connector with respect to M. All other connector elements that
are hierarchically inside M, but not in one of the outside connectors of
M, is called an inside connector with respect to M. This is done before
resolving outer elements to corresponding inner ones.

{[}\emph{Example:}
\begin{figure}[H]
\includegraphics[scale=0.95]{media/innerouterconnector}
\caption{Example for inside and outside connectors}
\end{figure}
\emph{The figure visualizes the following} connect \emph{equations to
the connector c in the models m\textit{i}. Consider the
following} \lstinline!connect! \emph{equations found in the model for component m0:}

\begin{lstlisting}[language=modelica]
  connect(m1.c, m3.c); // m1.c and m3.c are inside connectors
  connect(m2.c, m3.c); // m2.c and m3.c are inside connectors
\end{lstlisting}
\emph{and in the model for component m3 (c.x is a sub-connector inside
c):}

\begin{lstlisting}[language=modelica]
  connect(c, m4.c); // c is an outside
  connector, m4.c is an inside connector
  connect(c.x, m5.c); // c.x is an outside
  connector, m5.c is an inside connector
  connect(c , d) ; // c is an outside  connector, d  is an outside connector
\end{lstlisting}
\emph{and in the model for component m6:}

\begin{lstlisting}[language=modelica]
  connect(d, m7.c); // d is an outside connector, m7.c is an inside connector
\end{lstlisting}
{]}

\subsection{Expandable Connectors}\doublelabel{expandable-connectors}

If the expandable qualifier is present on a connector definition, all
instances of that connector are referred to as expandable connectors.
Instances of connectors that do not possess this qualifier will be
referred to as non-expandable connectors.

Before generating connection equations non-parameter scalar variables
and non-parameter array elements declared in expandable connectors are
marked as only being potentially present. A non-parameter array element
may be declared with array dimensions ``:'' indicating that the size is
unknown. This applies to both variables of simple types, and variables
of structured types.

Then connections containing expandable connectors are elaborated:

\begin{itemize}
\item
  One connector in the connect equation must reference a declared
  component, and if the other connector is an undeclared element in a
  declared expandable connector it is handled as follows (elements that
  are only potentially present are not seen as declared):
\item
  The expandable connector instance is automatically augmented with a
  new component having the used name and corresponding type.
\item
  If the undeclared component is subscripted, an array variable is
  created, and a connection to the specific array element is performed.
  Introducing elements in an array gives an array with at least the
  specified elements, other elements are either not created or have a
  default value (i.e. as if they were only potentially present).
\item
  If the variable on the other side of the connect-equation is input or
  output the new component will be either input or output to satisfy the
  restrictions in \ref{restrictions-of-connections-and-connectors} for a non-expandable connector.
  {[}\emph{If the existing side refers to an inside connector (i.e. a
  connector of a component) the new variable will copy its causality,
  i.e. input if input and output if output, since the expandable
  connector must be an outside connector}{]}. For an array the
  input/output property can be deduced separately for each array
  element.
\item
  When two expandable connectors are connected, each is augmented with
  the variables that are only declared in the other expandable connector
  (the new variables are neither input nor output). This is repeated
  until all connected expandable connector instances have matching
  variables {[}\emph{i.e. each of the connector instances is expanded to
  be the union of all connector variables}.{]}
\item
  The variables introduced in the elaboration follow additional rules
  for generating connection sets (given in \ref{generation-of-connection-equations}).
\item
  If a variable appears as an input in one expandable connector, it
  should appear as a non-input in at least one other expandable
  connector instance in the same augmentation set. An augmentation set
  is defined as the set of connected expandable connector instances that
  through the elaboration will have matching variables.

{[}\emph{Example}:
\begin{lstlisting}[language=modelica]
expandable connector EngineBus
end EngineBus;

block Sensor
  RealOutput speed; // Output, i.e., non-input
end Sensor;
block Actuator
  RealInput speed; // Input
end Actuator;

model Engine
  EngineBus bus;
  Sensor sensor;
  Actuator actuator;
equation
  connect(bus.speed, sensor.speed); // provides the non-input from sensor.speed
  connect(bus.speed, actuator.speed);
end Engine;
\end{lstlisting}
{]}

\item
  All components in an expandable connector are seen as connector
  instances even if they are not declared as such {[}\emph{i.e. it is
  possible to connect to e.g. a Real variable}{]}.

{[}\emph{Example}:
\begin{lstlisting}[language=modelica]
expandable connector EngineBus // has predefined signals
  import SI=Modelica.SIunits;
  SI.AngularVelocity speed;
  SI.Temperature T;
end EngineBus;

block Sensor
 RealOutput speed;
end Sensor;

model Engine
  EngineBus bus;
  Sensor sensor;
equation
  connect(bus.speed, sensor.speed);
  // connection to non-connector speed is possible
  // in expandable connectors
end Engine;
\end{lstlisting}
{]}

\item
  An expandable connector may not contain a component declared with the
  prefix flow, but may contain non-expandable connector components with
  flow components.

{[}\emph{Example}:
\begin{lstlisting}[language=modelica]
import Interfaces=Modelica.Electrical.Analog.Interfaces;
expandable connector ElectricalBus
  Interfaces.PositivePin p12, n12; // OK
  flow Modelica.SIunits.Current i; // not allowed
end ElectricalBus;

model Battery
  Interfaces.PositivePin p42, n42;
  ElectricalBus bus;
equation
  connect(p42, bus.p42); // Adds new electrical pin
  connect(n42, bus.n42); // Adds another pin
end Battery;
\end{lstlisting}

{]}

\item
  expandable connectors can only be connected to other expandable
  connectors.

  If a connect equation references a potentially present variable, or
  variable element, in an expandable connector the variable or variable
  element is marked as being present, and due to the paragraphs above it
  is possible to deduce whether the bus variable shall be treated as
  input, or shall be treated as output in the connect equation. That
  input or output prefix is added if no input/output prefix is present
  on the declaration

{[}\emph{Example}:
\begin{lstlisting}[language=modelica]
expandable connector EmptyBus
end EmptyBus;

model Controller
  EmptyBus bus1;
  EmptyBus bus2;
  RealInput speed;
equation
  connect(speed, bus1.speed); // ok, only one undeclared
  // and it is unsubscripted
  
  connect(bus1.pressure, bus2.pressure);
  // not allowed, both undeclared
  
  connect(speed, bus2.speed[2]);
  // introduces speed array (with element [2]).
end Controller;
\end{lstlisting}
{]}
\end{itemize}

After this elaboration the expandable connectors are treated as normal
connector instances, and the connections as normal connections, and all
potentially present variables and array elements that are not actually
present are undefined {[}\emph{a tool may remove them or set them to the
default value, e.g. zero for Real variables}{]}. It is an error if there
are expressions referring to potentially present variables or array
elements that are not actually present or non-declared variables
{[}\emph{the expressions can only ``read'' variables from the bus that
are actually declared and present in the connector, in order that the
types of the variables can be determined in the local scope}{]}. This
elaboration implies that expandable connectors can be connected even if
they do not contain the same components.

{[}\emph{Note that the introduction of variables, as described above, is
conceptual and does not necessarily impact the flattening hierarchy in
any way. Furthermore, it is important to note that these elaboration
rules must consider:}

\begin{enumerate}
\item \emph{Expandable connectors nested hierarchically. This means that
both outside and inside connectors must be included at every level of
the hierarchy in this elaboration process.}
\item \emph{When processing an expandable connector that possesses the}
inner \emph{scope qualifier, all outer instances must also be taken into
account during elaboration}.
\end{enumerate}
\emph{Example}:

\emph{Engine system with sensors, controllers, actuator and plant that
exchange information via a bus (i.e. via expandable connectors):}
\begin{lstlisting}[language=modelica]
import SI=Modelica.SIunits;
import Modelica.Blocks.Interfaces.RealInput;
// Plant Side
model SparkPlug
  RealInput spark_advance;
  ...
end SparkPlug;

expandable connector EngineBus
  // No minimal set
end EngineBus;

expandable connector CylinderBus
  Real spark_advance;
end CylinderBus;

model Cylinder
  CylinderBus cylinder_bus;
  SparkPlug spark_plug;
  ...
equation
  connect(spark_plug.spark_advance,
  cylinder_bus.spark_advance);
end Cylinder;

model I4
  EngineBus engine_bus;
  Modelica.Mechanics.Rotational.Sensors.SpeedSensor speed_sensor;
  Modelica.Thermal.HeatTransfer.Sensors.TemperatureSensor temp_sensor;
  parameter Integer nCylinder = 4 "Number of cylinders";
  Cylinder cylinder[nCylinder];
equation
  // adds engine_speed (as output)
  connect(speed_sensor.w, engine_bus.engine_speed);
  // adds engine_temp (as output)
  connect(temp_sensor.T, engine_bus.engine_temp);
  // adds cylinder_bus1 (a nested bus)
  for i in 1:nCylinder loop
    connect(cylinder[i].cylinder_bus,
    engine_bus.cylinder_bus[i]);
  end for;
end I4;
\end{lstlisting}
\emph{Due to the above connection, conceptually a connector consisting
of the union of all connectors is introduced.}

\emph{The engine\_bus contains the following variable declarations:}
\begin{lstlisting}[language=modelica]
  RealOutput engine_speed;
  RealOutput engine_temp;
  CylinderBus cylinder_bus[1];
  CylinderBus cylinder_bus[2];
  CylinderBus cylinder_bus[3];
  CylinderBus cylinder_bus[4];
\end{lstlisting}
{]}

\section{Generation of Connection Equations}\doublelabel{generation-of-connection-equations}

When generating connection equations, outer elements are resolved to the
corresponding inner elements in the instance hierarchy (see instance
hierarchy name lookup \ref{instance-hierarchy-name-lookup-of-inner-declarations}). The arguments to each connect-equation are
resolved to two connector elements.

For every use of the connect-equation
\begin{lstlisting}[language=modelica]
connect(a, b);
\end{lstlisting}

the primitive components of a and b form a connection set -- together
with an indication of whether they are from an inside or an outside
connector; the primitive elements are of simple types -- or of types
defined as operator record (i.e. a component of an operator record type
is not split into sub-components). The elements of the connection sets
are tuples of primitive variables together with an indication of inside
or outside; if the same tuple belongs to two connection sets those two
sets are merged, until every tuple is only present in one set. Composite
connector types are broken down into primitive components. The outer
components are handled by mapping the objects to the corresponding inner
components -- and the inside indication is not influenced. The outer
connectors are handled by mapping the objects to the corresponding inner
connectors -- and they are always treated as outside connectors.

{[}\emph{Rationale: The inside/outside as part of the connection sets
ensure that connections from different hierarchical levels are treated
separately. Connection sets are formed from the primitive elements and
not from the connectors; this handles connections to parts of
hierarchical connectors and also makes it easier to generate equations
directly from the connection sets. All variables in one connection set
will either be flow variables or non-flow variables due to restriction
on connect-equations. The mapping from an} outer \emph{to an} inner
\emph{element must occur before merging the sets in order to get one
zero-sum equation, and ensures that the equations for the} outer
\emph{elements are all given for ``one side'' of the connector, and the}
inner \emph{element can define the other ``side''.}{]}

The following connection sets with just one member are also present (and
merged):

\begin{itemize}
\item
  Each primitive flow-variable as inside connector.
\item
  Each flow variable \emph{added} during augmentation of expandable
  connectors, both as inside and as outside. \emph{{[}Note that the flow
  variable is not directly in the expandable connector, but in a
  connector inside the expandable connector.{]}}
\end{itemize}

{[}\emph{Rationale: If these variables are not connected they will
generate a set comprised only of this element, and thus they will be
implicitly set to zero (see below). If connected, this set will be
merged and adding this at the start has no impact.}{]}

Each connection set is used to generate equations for potential and flow
(zero-sum) variables of the form

\begin{itemize}
\item
\lstinline!a1 = a2 = ... = an; // neither flow nor stream variables!
\item
\lstinline!z1 + z2 + (-z3) + ... + zn =! \textbf{0} \lstinline!; // flow-variables!
\end{itemize}

The bold-face \textbf{0} represents an array or scalar zero of
appropriate dimensions (i.e. the same size as z).

For an operator record type this uses the operator '0' -- which must be
defined in the operator record; and all of the flow-variables for the
operator record must be of the same operator record type. This implies
that in order to have flow variables of an operator record type the
operator record must define addition, negation, and '0'; and these
operations should define an additive group.

In order to generate equations for flow variables {[}\emph{using the}
\textbf{flow} \emph{prefix}{]}, the sign used for the connector variable
$z_i$ above is +1 for inside connectors and -1 for outside
connectors $z_3$ \emph{in the example above}{]}.

{[}\emph{Example (simple):}

\begin{lstlisting}[language=modelica]
model Circuit
  Ground ground;
  Load load;
  Resistor resistor;
equation
  connect(load.p , ground.p);
  connect(resistor.p, ground.p);
end Circuit;

model Load
  extends TwoPin;
  Resistor resistor;
equation
  connect(p, resistor.p);
  connect(resistor.n, n);
end Load;
\end{lstlisting}
\emph{The connection sets are before merging (note that one part of the
load and resistor is not connected):}

\emph{\{\textless{}load.p.i, inside\textgreater{}\}}

\emph{\{\textless{}load.n.i, inside\textgreater{}\}}

\emph{\{\textless{}ground.p.i, inside\textgreater{}\}}

\emph{\{\textless{}load.resistor.p.i, inside\textgreater{}\}}

\emph{\{\textless{}load.resistor.n.i, inside\textgreater{}\}}

\emph{\{\textless{}resistor.p.i, inside\textgreater{}\}}

\emph{\{\textless{}resistor.n.i, inside\textgreater{}\}}

\emph{\{\textless{}resistor.p.i, inside\textgreater{},
\textless{}ground.p.i, inside\textgreater{}\}}

\emph{\{\textless{}resistor.p.v, inside\textgreater{},
\textless{}ground.p.v, inside\textgreater{}\}}

\emph{\{\textless{}load.p.i, inside\textgreater{},
\textless{}ground.p.i, inside\textgreater{}\}}

\emph{\{\textless{}load.p.v, inside\textgreater{},
\textless{}ground.p.v, inside\textgreater{}\}}

\emph{\{\textless{}load.p.i, outside\textgreater{},
\textless{}load.resistor.p.i, inside\textgreater{}\}}

\emph{\{\textless{}load.p.v, outside\textgreater{},
\textless{}load.resistor.p.v, inside\textgreater{}\}}

\emph{\{\textless{}load.n.i, outside\textgreater{},
\textless{}load.resistor.n.i, inside\textgreater{}\}}

\emph{\{\textless{}load.n.v, outside\textgreater{},
\textless{}load.resistor.n.v, inside\textgreater{}\}}

\emph{After merging this gives:}

\emph{\{\textless{}load.p.i, outside\textgreater{},
\textless{}load.resistor.p.i, inside\textgreater{}\}}

\emph{\{\textless{}load.p.v, outside\textgreater{},
\textless{}load.resistor.p.v, inside\textgreater{}\}}

\emph{\{\textless{}load.n.i, outside\textgreater{},
\textless{}load.resistor.n.i, inside\textgreater{}\}}

\emph{\{\textless{}load.n.v, outside\textgreater{},
\textless{}load.resistor.n.v, inside\textgreater{}\}}

\emph{\{\textless{}load.p.i, inside\textgreater{},
\textless{}ground.p.i, inside\textgreater{}, \textless{}resistor.p.i,
inside\textgreater{} \}}

\emph{\{\textless{}load.p.v, inside\textgreater{},
\textless{}ground.p.v, inside\textgreater{}, \textless{}resistor.p.v,
inside\textgreater{}\}}

\emph{\{\textless{}load.n.i, inside\textgreater{}\}}

\emph{\{\textless{}resistor.n.i, inside\textgreater{}\}}

\emph{And thus the equations:}

load.p.v = load.resistor.p.v;

load.n.v = load.resistor.n.v;

load.p.v = ground.p.v;

load.p.v = resistor.p.v;

0 = (-load.p.i) + load.resistor.p.i;

0 = (-load.n.i) + load.resistor.n.i;

0 = load.p.i + ground.p.i + resistor.p.i;

0 = load.n.i;

0 = resistor.n.i;

\emph{Example (outer component):}

\begin{lstlisting}[language=modelica]
model Circuit
  Ground ground;
  Load load;
  inner Resistor resistor;
equation
  connect(load.p, ground.p);
end Circuit;

model Load
  extends TwoPin;
  outer Resistor resistor;
equation
  connect(p, resistor.p);
  connect(resistor.n, n);
end Load;
\end{lstlisting}
\emph{The connection sets are before merging (note that one part of the
load and resistor is not connected):}

\emph{\{\textless{}load.p.i, inside\textgreater{}\}}

\emph{\{\textless{}load.n.i, inside\textgreater{}\}}

\emph{\{\textless{}ground.p.i, inside\textgreater{}\}}

\emph{\{\textless{}resistor.p.i, inside\textgreater{}\}}

\emph{\{\textless{}resistor.n.i, inside\textgreater{}\}}

\emph{\{\textless{}load.p.i, inside\textgreater{},
\textless{}ground.p.i, inside\textgreater{}\}}

\emph{\{\textless{}load.p.v, inside\textgreater{},
\textless{}ground.p.v, inside\textgreater{}\}}

\emph{\{\textless{}load.p.i, outside\textgreater{}, \textless{}
resistor.p.i, inside\textgreater{}\}}

\emph{\{\textless{}load.p.v, outside\textgreater{},
\textless{}resistor.p.v, inside\textgreater{}\}}

\emph{\{\textless{}load.n.i, outside\textgreater{},
\textless{}resistor.n.i, inside\textgreater{}\}}

\emph{\{\textless{}load.n.v, outside\textgreater{},
\textless{}resistor.n.v, inside\textgreater{}\}}

\emph{After merging this gives:}

\emph{\{\textless{}load.p.i, outside\textgreater{},
\textless{}resistor.p.i, inside\textgreater{}\}}

\emph{\{\textless{}load.p.v, outside\textgreater{},
\textless{}resistor.p.v, inside\textgreater{}\}}

\emph{\{\textless{}load.n.i, outside\textgreater{},
\textless{}resistor.n.i, inside\textgreater{}\}}

\emph{\{\textless{}load.n.v, outside\textgreater{},
\textless{}resistor.n.v, inside\textgreater{}\}}

\emph{\{\textless{}load.p.i, inside\textgreater{},
\textless{}ground.p.i, inside\textgreater{}\}}

\emph{\{\textless{}load.p.v, inside\textgreater{},
\textless{}ground.p.v, inside\textgreater{}\}}

\emph{\{\textless{}load.n.i, inside\textgreater{}\}}

\emph{And thus the equations:}

load.p.v = resistor.p.v;

load.n.v = resistor.n.v;

load.p.v = ground.p.v;

0 = (-load.p.i) + resistor.p.i;

0 = (-load.n.i) + resistor.n.i;

0 = load.p.i + ground.p.i;

0 = load.n.i;

\emph{This corresponds to a direct connection of the resistor.}

{]}

\section{Restrictions of Connections and Connectors}\doublelabel{restrictions-of-connections-and-connectors}

\begin{itemize}
\item
  The connect-equations (and the special functions for overdetermined
  connectors) may only be used in equations and may not be used inside
  if-equations with non-parametric condition, or in when-equations.
  {[}\emph{For-equations always have parameter expressions for the array
  expression}.{]}
\item
  A connector component may not be declared with the prefix parameter or
  constant. In the connect-equation the primitive components may only
  connect parameter variables to parameter variables and constant
  variables to constant variables.
\item
  The connect-equation construct only accepts forms of connector
  references as specified in \ref{connect-equations-and-connectors}.
\item
  In a connect-equation the two connectors must have the same named
  component elements with the same dimensions; recursively down to the
  primitive components. The primitive components with the same name are
  matched and belong to the same connection set.
\item
  The matched primitive components of the two connectors must have the
  same primitive types, and flow-variables may only connect to other
  flow-variables, stream-variables only to other stream-variables, and
  causal variables (input/output) only to causal variables
  (input/output).
\item
  A connection set of causal variables (input/output) may at most
  contain variables from one inside output connector or one public
  outside input connector. {[}\emph{i.e., a connection set may at most
  contain one source of a signal.}{]}
\item
  At least one of the following must hold for a connection set
  containing causal variables generated for a non-partial model or
  block:
\begin{enumerate}
\item the connection set includes variables from an outside public
  expandable connector, 
\item the set contains variables from protected
  outside connectors, 
\item it contains variables from one inside output
  connector, or 
\item from one public outside input connector, or 
\item the  set is comprised solely of one variable from one inside input
  connector that is not part of an expandable connector. \label{exc-conn-case}
\end{enumerate}
{[}\emph{i.e., a connection set must -- unless the model or block is partial -
  contain one source of a signal (the last item (\ref{exc-conn-case}) covers the case
  where a connector of a component is left unconnected and the source
  given textually).}{]}
\item
  Variables from a protected outside connector must be part of a
  connection set containing at least one inside connector or one
  declared public outside connector (i.e. it may not be an implicitly
  defined part of an expandable connector). {[}\emph{Otherwise it would
  not be possible to deduce the causality for the expandable connector
  element.}{]}
\item
  In a connection set all variables having non-empty quantity attribute
  must have the same quantity attribute.
\item
  A connect equation may not (directly or indirectly) connect two
  connectors of outer elements. {[}\emph{indirectly is similar to them
  being part of the same connection set -- however, connections to}
  outer \emph{elements are ``moved up'' before forming connection sets.
  Otherwise the connection sets could contain ``redundant'' information
  breaking the equation count for locally balanced models and
  blocks.}{]}
\item
  Subscripts in a connector reference shall be parameter expressions or
  the special operator ``:''.
\item
  Constants or parameters in connected components yield the appropriate
  assert statements to check that they have the same value; connections
  are not generated.
\item
  For conditional connectors, see \ref{conditional-component-declaration}.
\end{itemize}

\subsection{Balancing Restriction and Size of Connectors}\doublelabel{balancing-restriction-and-size-of-connectors}

For each non-partial connector class the number of flow variables shall
be equal to the number of variables that are neither parameter,
constant, input, output, stream nor flow. The ``number of variables'' is
the number of all elements in the connector class after expanding all
records and arrays to a set of scalars of primitive types. The number of
variables of an overdetermined type or record class (see \ref{overconstrained-equation-operators-for-connection-graphs})
is the size of the output argument of the corresponding
equalityConstraint() function.

{[}\emph{Examples:}

\begin{lstlisting}[language=modelica]
connector Pin // a physical connector of
  Modelica.Electrical.Analog
  Real v;
  flow Real i;
end Pin;

connector Plug // a hierarchical connector of
  Modelica.Electrical.MultiPhase
  parameter Integer m=3;
  Pin p[m];
end Plug;

connector InputReal = input Real; // A causal input connector
connector OutputReal = output Real; // A causal output connector

connector Frame_Illegal
  Modelica.SIunits.Position r0[3] "Position vector of frame origin";
  Real S[3, 3] "Rotation matrix of frame";
  Modelica.SIunits.Velocity v[3] "Abs. velocity of frame origin";
  Modelica.SIunits.AngularVelocity w[3] "Abs. angular velocity of frame";
  Modelica.SIunits.Acceleration a[3] "Abs. acc. of frame origin";
  Modelica.SIunits.AngularAcceleration z[3] "Abs. angular acc. of frame";
  flow Modelica.SIunits.Force f[3] "Cut force";
  flow Modelica.SIunits.Torque t[3] "Cut torque";
end Frame_Illegal;
\end{lstlisting}

\emph{The} Frame\_Illegal \emph{connector (intended to be used in a
simple MultiBody-package without over-determined connectors) is illegal
since the number of flow and non-flow variables do not match. The
solution is to create two connector classes, where two 3-vectors (e.g.,
a and z) are acausal Real and the other variables are matching pairs of}
input \emph{and} output\emph{. This ensures that the models can only be
connected in a tree-structure or require a ``loop-breaker'' joint for
every closed kinematic loop:}

\begin{lstlisting}[language=modelica]
connector Frame_a "correct connector"
  input Modelica.SIunits.Position r0[3];
  input Real S[3, 3];
  input Modelica.SIunits.Velocity v[3];
  input Modelica.SIunits.AngularVelocity w[3];
  Modelica.SIunits.Acceleration a[3];
  Modelica.SIunits.AngularAcceleration z[3];
  flow Modelica.SIunits.Force f[3];
  flow Modelica.SIunits.Torque t[3];
end Frame_a;

connector Frame_b "correct connector"
  output Modelica.SIunits.Position r0[3];
  output Real S[3, 3];
  output Modelica.SIunits.Velocity v[3];
  output Modelica.SIunits.AngularVelocity w[3];
  Modelica.SIunits.Acceleration a[3];
  Modelica.SIunits.AngularAcceleration z[3];
  flow Modelica.SIunits.Force f[3];
  flow Modelica.SIunits.Torque t[3];
end Frame_b;
\end{lstlisting}

\emph{The subsequent connectors} Plug\_Expanded \emph{and} PlugExpanded2
\emph{are correct, but} Plug\_Expanded\_Illegal \emph{is illegal since
the number of non-flow and flow variables is different if ``}n\emph{''
and ``}m\emph{'' are different. It is not clear how a tool can detect in
general that connectors such as} Plug\_Expanded\_Illegal \emph{are
illegal. However, it is always possible to detect this defect after
actual values of parameters and constants are provided in the simulation
model.}

\begin{lstlisting}[language=modelica]
connector Plug_Expanded "correct connector"
  parameter Integer m=3;
  Real v[m];
  flow Real i[m];
end Plug_Expanded;

connector Plug_Expanded2 "correct connector"
  parameter Integer m=3;
  final parameter Integer n=m;
  Real v[m];
  flow Real i[n];
end Plug_Expanded2;

connector Plug_Expanded_Illegal "connector is illegal"
  parameter Integer m=3;
  parameter Integer n=m;
  Real v[m];
  flow Real i[n];
end Plug_Expanded_Illegal;
\end{lstlisting}

{]}

\section{Equation Operators for Overconstrained Connection-Based Equation Systems}\doublelabel{equation-operators-for-overconstrained-connection-based-equation-systems1}

There is a special problem regarding equation systems resulting from
\emph{loops} in connection graphs where the connectors contain
\emph{non-flow} (i.e., potential) variables \emph{dependent} on each
other. When a loop structure occurs in such a graph, the resulting
equation system will be \emph{overconstrained}, i.e., have more
equations than variables, since there are implicit constraints between
certain non-flow variables in the connector in addition to the
connection equations around the loop. At the current state-of-the-art,
it is not possible to automatically eliminate the unneeded equations
from the resulting equation system without additional information from
the model designer.

This section describes a set of equation operators for such
overconstrained connection-based equation systems, that makes it
possible for the model designer to specify enough information in the
model to allow a Modelica environment to automatically remove the
superfluous equations.

{[}\emph{Connectors may contain redundant variables. For example, the
orientation between two coordinate systems in 3 dimensions can be
described by 3 independent variables. However, every description of
orientation with 3 variables has at least one singularity in the region
where the variables are defined. It is therefore not possible to declare
only 3 variables in a connector. Instead n variables (n \textgreater{}
3) have to be used. These variables are no longer independent from each
other and there are n-3 constraint equations that have to be fulfilled.
A proper description of a redundant set of variables with constraint
equations does no longer have a singularity. A model that has loops in
the connection structure formed by components and connectors with
redundant variables, may lead to a differential algebraic equation
system that has more equations than unknown variables. The superfluous
equations are usually consistent with the rest of the equations, i.e., a
unique mathematical solution exists. Such models cannot be treated with
the currently known symbolic transformation methods. To overcome this
situation, operators are defined in order that a Modelica translator can
remove the superfluous equations. This is performed by replacing the
equality equations of non-flow variables from connection sets by a
reduced number of equations in certain situations.}

\emph{This section handles a certain class of overdetermined systems due
to connectors that have a redundant set of variables. There are other
causes of overdetermined systems, e.g., explicit zero-sum equations for
flow variables, that are not handled by the method described below}.{]}

\subsection{Overconstrained Equation Operators for Connection Graphs}\doublelabel{overconstrained-equation-operators-for-connection-graphs}

A type or record declaration may have an optional definition of function
``equalityConstraint(..)'' that shall have the following prototype:

\begin{lstlisting}[language=modelica]
type Type // overdetermined type
  extends <base type>;
  function equalityConstraint // non-redundant equality
    input Type T1;
    input Type T2;
    output Real residue[ <n> ];
  algorithm
    residue := ...
  end equalityConstraint;
end Type;

record Record
  < declaration of record fields>
  function equalityConstraint // non-redundant equality
    input Record R1;
    input Record R2;
    output Real residue[ <n> ];
  algorithm
    residue := ...
  end equalityConstraint;
end Record;
\end{lstlisting}
The ``residue'' output of the equalityConstraint(..) function shall have
known size, say constant n. The function shall express the equality
between the two type instances T1 and T2 or the record instances R1 and
R2, respectively, with a non-redundant number $ n \ge 0$ of equations. The
residues of these equations are returned in vector ``residue'' of size
n. The set of n non-redundant equations stating that R1 = R2 is given by
the equation (\textbf{0} characterizes a vector of zeros of appropriate
size):

\begin{lstlisting}[language=modelica]
  Record R1, R2;
equation
  0 = Record.equalityConstraint(R1,R2);
\end{lstlisting}
{[}\emph{If the elements of a record} Record \emph{are not independent
from each other, the equation ``}R1 = R2\emph{'' contains redundant
equations}{]}.

A type class with an equalityConstraint function declaration is called
overdetermined type. A record class with an equalityConstraint function
definition is called overdetermined record. A connector that contains
instances of overdetermined type and/or record classes is called
overdetermined connector. An overdetermined type or record may neither
have flow components nor may be used as a type of flow components. If an
array is used as argument to any of the Connections.* functions it is
treated as one unit -- there is no special treatment of this case --
however, there is for connect -- see \ref{connect-equations-and-connectors}.

Every instance of an overdetermined type or record in an overdetermined
connector is a node in a virtual connection graph that is used to
determine when the standard equation ``R1 = R2'' or when the equation
``0 = equalityConstraint(R1,R2)''has to be used for the generation of
connect(...) equations. The branches of the virtual connection graph are
implicitly defined by ``connect(..)'' and explicitly by
Connections.branch(...) statements, see table below. Connections is a
built-in package in global scope containing built-in operators.
Additionally, corresponding nodes of the virtual connection graph have
to be defined as roots or as potential roots with functions
Connections.root(...) and Connections.potentialRoot(...), respectively.
In the following table, A and B are connector instances that may be
hierarchically structured, e.g., A may be an abbreviation for
EnginePort.Frame.

\begin{longtable}[]{|p{5.1cm}|p{10cm}|}
\hline \endhead
connect(A,B); & Defines \emph{breakable branches} from the
overdetermined type or record instances in connector instance A to the
corresponding overdetermined type or record instances in connector
instance B for a virtual connection graph. The types of the
corresponding overdetermined type or record instances shall be the
same.\\ \hline
Connections.branch(A.R,B.R); & Defines a \emph{non-breakable branch}
from the overdetermined type or record instance R in connector instance
A to the corresponding overdetermined type or record instance R in
connector instance B for a virtual connection graph. This function can
be used at all places where a connect(..) statement is allowed
{[}\emph{e.g., it is not allowed to use this function in a when-clause.
This definition shall be used if in a model with connectors A and B the
overdetermined records} A.R \emph{and} B.R \emph{are algebraically
coupled in the model, e.g., due to} B.R = f(A.R\emph{, \textless{}other
unknowns\textgreater{})}{]}\emph{.}\\ \hline
Connections.root(A.R); & The overdetermined type or record instance R in
connector instance A is a (definite) \emph{root node} in a virtual
connection graph. {[}\emph{This definition shall be used if in a model
with connector} A \emph{the overdetermined record} A.R \emph{is
(consistently) assigned, e.g., from a parameter
expressions}{]}\\ \hline
\begin{tabular}{@{}p{5.1cm}@{}}
Connections.potentialRoot(A.R);\\
Connections.potentialRoot(\\
A.R, priority = p);
\end{tabular}
& The overdetermined type or record instance R in connector instance A is
a \emph{potential root node} in a virtual connection graph with priority
``p'' ($p\ge 0$). If no second argument is provided, the priority is zero.
``p'' shall be a parameter expression of type Integer\emph{.} In a
virtual connection subgraph without a Connections.root definition, one
of the potential roots with the lowest priority number is selected as
root {[}\emph{This definition may be used if in a model with connector}
A \emph{the overdetermined record} A.R \emph{appears differentiated --}
der(A.R) \emph{-- together with the} constraint equations \emph{of}
A.R\emph{, i.e., a non-redundant subset of} A.R \emph{maybe used as
states}{]}
%\strut
%\end{minipage}
\\ \hline
b = Connections.isRoot(A.R); & Returns true, if the overdetermined type
or record instance R in connector instance A is selected as a root in
the virtual connection graph.\\ \hline
\begin{tabular}{@{}p{5.1cm}@{}}
b = Connections.rooted(A.R);\\
b =rooted(A.R); // deprecated
\end{tabular}
& If the operator Connections.rooted(A.R)
is used, or the equivalent but deprecated operator rooted(A.R), then
there must be exactly one statement Connections.branch(A.R,B.R)
involving A.R (the argument of Connections.rooted must be the first
argument of Connections.branch). In that case Connections.rooted(A.R)
returns true, if A.R is closer to the root of the spanning tree than
B.R; otherwise false is returned. {[}\emph{This operator can be used to
avoid equation systems by providing analytic inverses, see
Modelica.Mechanics.MultiBody.Parts.FixedRotation.}{]}\\ \hline
\end{longtable}

{[}\emph{Note, that} Connections.branch\emph{,} Connections.root\emph{,}
Connections.potentialRoot \emph{do not generate equations. They only
generate nodes and branches in the virtual graph for analysis
purposes.}{]}

\subsection{Converting the Connection Graph into Trees and Generating Connection Equations}\doublelabel{converting-the-connection-graph-into-trees-and-generating-connection-equations}

Before connect(...) equations are generated, the virtual connection
graph is transformed into a set of spanning trees by removing breakable
branches from the graph. This is performed in the following way:

\begin{enumerate}
\item
  Every root node defined via the ``Connections.root(..)'' statement is
  a definite root of one spanning tree.
\item
  The virtual connection graph may consist of sets of subgraphs that are
  not connected together. Every subgraph in this set shall have at least
  one root node or one potential root node in a simulation model. If a
  graph of this set does not contain any root node, then one potential
  root node in this subgraph that has the lowest priority number is
  selected to be the root of that subgraph. The selection can be
  inquired in a class with function Connections.isRoot(..), see table
  above.
\item
  If there are n selected roots in a subgraph, then breakable branches
  have to be removed such that the result shall be a set of n spanning
  trees with the selected root nodes as roots.
\end{enumerate}

After this analysis, the connection equations are generated in the
following way:

\begin{enumerate}
\item
  For every breakable branch {[}\emph{i.e., a} connect(A,B)
  \emph{equation,}{]} in one of the spanning trees, the connection
  equations are generated according to \ref{generation-of-connection-equations}.
\item
  For every breakable branch not in any of the spanning trees, the
  connection equations are generated according to \ref{generation-of-connection-equations}, except
  for overdetermined type or record instances R. Here the equations
  ``\textbf{0} = R.equalityConstraint(A.R,B.R)'' are generated instead
  of ``A.R = B.R''.
\end{enumerate}

\subsection{Examples of Overconstrained Connection Graphs}\doublelabel{examples-of-overconstrained-connection-graphs}

{[}\emph{Example:}

\begin{figure}[H]
\caption{Example of a virtual connection graph.}
% Need to shrink a little to avoid overully
\includegraphics[scale=0.85]{media/overdetermined}
\end{figure}

{]}

\subsubsection{An Overdetermined Connector for Power Systems}\doublelabel{an-overdetermined-connector-for-power-systems}

{[}\emph{An overdetermined connector for power systems based on the
transformation theory of Park may be defined as:}

\begin{lstlisting}[language=modelica]
type AC_Angle "Angle of source, e.g., rotor of generator"
  extends Modelica.SIunits.Angle; // AC_Angle is a Real number
  // with unit = "rad"
  function equalityConstraint
    input AC_Angle theta1;
    input AC_Angle theta2;
    output Real residue[0] "No constraints"
  algorithm
    /* make sure that theta1 and theta2 from joining branches are identical */
    assert(abs(theta1 -- theta2) < 1.e-10, "Consistent angles");
  end equalityConstraint;
end AC_Angle;

connector AC_Plug "3-phase alternating current connector"
  import SI = Modelica.SIunits;
  AC_Angle theta;
  SI.Voltage v[3] "Voltages resolved in AC_Angle frame";
  flow SI.Current i[3] "Currents resolved in AC_Angle
frame";
end AC_Plug;
\end{lstlisting}
\emph{The currents and voltages in the connector are defined relatively
to the harmonic, high-frequency signal of a power source that is
essentially described by angle theta of the rotor of the source. This
allows much faster simulations, since the basic high frequency signal of
the power source is not part of the differential equations. For example,
when the source and the rest of the line operates with constant
frequency (= nominal case), then} AC\_Plug.v \emph{and} AC\_Plug.i
\emph{are constant. In this case a variable step integrator can select
large time steps. An element, such as a 3-phase inductor, may be
implemented as:}

\begin{lstlisting}[language=modelica]
model AC_Inductor
  parameter Real X[3,3], Y[3,3]; // component constants
  AC_plug p;
  AC_plug n;
equation
  Connections.branch(p.theta,n.theta); //branch in virtual graph
  // since n.theta = p.theta
  n.theta = p.theta; // pass angle theta between plugs
  omega = der (p.theta); // frequency of source
  zeros(3) = p.i + n.i;
  X*der (p.i) + omega*Y*p.i = p.v -- n.v;
end AC_Inductor
\end{lstlisting}
\emph{At the place where the source frequency, i.e., essentially
variable theta, is defined, a} Connections.root(..) \emph{must be
present:}

\begin{lstlisting}[language=modelica]
  AC_plug p;
equation
  Connections.root(p.theta);
  der(p.theta) = 2*Modelica.Constants.pi*50 // 50 Hz;
\end{lstlisting}
\emph{The graph analysis performed with the virtual connection graph
identifies the connectors, where the} AC\_Angle \emph{needs not to be
passed between components, in order to avoid redundant equations.}

\subsubsection{An Overdetermined Connector for 3-dimensional Mechanical Systems}\doublelabel{an-overdetermined-connector-for-3-dimensional-mechanical-systems}

\emph{An overdetermined connector for 3-dimensional mechanical systems
may be defined as:}

\begin{lstlisting}[language=modelica]
  type TransformationMatrix = Real[3,3];
  type Orientation "Orientation from frame 1 to frame 2"
    extendsTransformationMatrix;
    function equalityConstraint
      input Orientation R1 "Rotation from inertial frame to frame 1";
      input Orientation R2 "Rotation from inertial frame to frame 2";
      output Real residue[3];
      protected
      Orientation R_rel "Relative Rotation from frame 1 to frame 2";
    algorithm
      R_rel = R2*transpose(R1);
      /* If frame_1 and frame_2 are identical, R_rel must be
the unit matrix. If they are close together, R_rel can be
linearized yielding:
R_rel = [ 1, phi3, -phi2;
-phi3, 1, phi1;
phi2, -phi1, 1 ];
where phi1, phi2, phi3 are the small rotation angles around
axis x, y, z of frame 1 to rotate frame 1 into frame 2.
The atan2 is used to handle large rotation angles, but does not
modify the result for small angles.
*/
      residue := { Modelica.Math.atan2(R_rel[2, 3], R_rel[1, 1]),
      Modelica.Math.atan2(R_rel[3, 1], R_rel[2, 2]),
      Modelica.Math.atan2(R_rel[1, 2], R_rel[3, 3])};
    end equalityConstraint;
  end Orientation;

  connector Frame "3-dimensional mechanical connector"
    import SI = Modelica.SIunits;
    SI.Position r[3] "Vector from inertial frame to Frame";
    Orientation R "Orientation from inertial frame to Frame";
    flow SI.Force f[3] "Cut-force resolved in Frame";
    flow SI.Torque t[3] "Cut-torque resolved in Frame";
  end Frame;
\end{lstlisting}
\emph{A fixed translation from a frame} A \emph{to a frame} B \emph{may
be defined as}:

\begin{lstlisting}[language=modelica]
model FixedTranslation
  parameter Modelica.SIunits.Position r[3];
  Frame frame_a, frame_b;
equation
  Connections.branch(frame_a.R, frame_b.R);
  frame_b.r = frame_a.r + transpose(frame_a.R)*r;
  frame_b.R = frame_a.R;
  zeros(3) = frame_a.f + frame_b.f;
  zeros(3) = frame_a.t + frame_b.t + cross(r, frame_b.f);
end FixedTranslation;
\end{lstlisting}
\emph{Since the transformation matrix} frame\_a.R \emph{is algebraically
coupled with} frame\_b.R\emph{, a branch in the virtual connection graph
has to be defined. At the inertial system, the orientation is
consistently initialized and therefore the orientation in the inertial
system connector has to be defined as root}:

\begin{lstlisting}[language=modelica]
model InertialSystem
  Frame frame_b;
equation
  Connections.root(frame_b.R);
  frame_b.r = zeros(3);
  frame_b.R = identity(3);
end InertialSystem;
\end{lstlisting}
{]}
